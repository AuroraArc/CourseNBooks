\documentclass[12pt,a4paper]{article}
\usepackage[utf8]{inputenc}
\usepackage[english]{babel}
\usepackage{amsmath}
\usepackage{amsfonts}
\usepackage{amssymb}
\usepackage{graphicx}
\usepackage[left=2cm,right=2cm,top=2cm,bottom=2cm]{geometry}
\author{Henry Yu}
\begin{document}

\raggedright

\begin{center}
	\huge{Note 1}
\end{center}

\section*{Propositional Logic}

To become fluent in working with mathematical statements, you need to understand the basic framework of the language of mathematics.

\bigbreak

\textbf{Proposition}: a statement which is either true or false

\bigbreak

These statements are propositions:
\begin{itemize}
	\item[1.] $\sqrt{3}$ is irrational.
	\item[2.]$1+1=5$.
\end{itemize}

These statements are not propositions:
\begin{itemize}
	\item[1.] $2+2$.
	\item[2.]$x^2+3x=5$. (since x is unknown)
\end{itemize}

Propositions should not include fuzzy terms, so these statements aren't propositions either (although some sources may say they are):
\begin{itemize}
	\item[1.] Arnold Schwarzenegger often eats broccoli. ("often" is fuzzy)
	\item[2.] Henry VIII was unpopular. ("unpopular" is fuzzy)
\end{itemize}

Propositions may be joined together to form more complex statements. Let $P$, $Q$, \text{and } $R$ be variables representing propositions. The simplest way of joining these propositions together is to use the connectives "and," "or," and "not."
\begin{itemize}
	\item[1.] \textbf{Conjunction}: $P\wedge Q$ ("$P$ and $Q$"). True only when both $P$ and $Q$ are true.
	\item[2.] \textbf{Disjunction}: $P\vee Q$ ("$P$ or $Q$"). True when at least one of $P$ and $Q$ is true.
	\item[3.] \textbf{Negation}: $\neg P$ ("not $P$"). True when $P$ is false.
\end{itemize}

Statements with variables (like these) are called \textbf{propositional forms}.

\bigbreak

A fundamental principle known as the \textbf{law of the excluded middle} says that, for any proposition $P$, either $P$ is true or $\neg P$ is true (but not both). Thus $P\vee\neg P$ is always true, regardless of the truth value of $P$. A propositional form is always true regardless of its truth values is called a \textbf{tautology}. Conversely, a statement such as $P\wedge\neg P$, which is always false, is called a \textbf{contradiction}.

\bigbreak

A
\end{document}
\documentclass[12pt,a4paper]{article}
\usepackage[utf8]{inputenc}
\usepackage[english]{babel}
\usepackage{amsmath}
\usepackage{amsfonts}
\usepackage{amssymb}
\usepackage{graphicx}
\usepackage[left=2cm,right=2cm,top=2cm,bottom=2cm]{geometry}
\begin{document}

\raggedright

\begin{center}
	\huge{Note 2}
\end{center}

\section*{Important Things}

\subsection*{Theorems}

\textbf{Theorem 2.1.} $\textit{For any } a,b,c \in \mathbb{Z}, \textit{if } a\mid b \textit{ and } a\mid c, \textit{then } a\mid (b+c)$.

\bigbreak

\textbf{Theorem 2.2.} \textit{Let} $0<n<1000$ \textit{be an integer. If the sum of the digits of n is divisible by 9, then n is divisible by 9}.

\bigbreak

\textbf{Theorem 2.3} (Converse of Theorem 2.2)\textbf{.} \textit{Let} $0<n<1000$ \textit{be an integer. If $n$ is divisible by 9, then the sum of the digits of $n$ is divisible by 9}.

\bigbreak

\textbf{Theorem 2.4.} \textit{Let $n$ be a positive integer and let $d$ divide $n$. If $n$ is odd, then $d$ is odd}.

\bigbreak

\textbf{Theorem 2.5} (Pigeonhole principle)\textit{.} \textit{Let $n$ and $k$ be positive integers. Place $n$ objects into $k$ boxes. If $n>k$, the at least one box must contain multiple objects}.

\bigbreak

\textbf{Theorem 2.6.} \textit{There are infinitely many prime numbers}.

\bigbreak

\textbf{Theorem 2.7.} \textit{$\sqrt{2}$ is irrational}.

\subsection*{Lemmas}

\textbf{Lemma 2.1.} \textit{Every natural number greater than one is either prime or has a prime divisor}.

\bigbreak

\textbf{Lemma 2.2.} \textit{If $a^2$ is even, then $a$ is even}.

\newpage

\section*{When to Use What}

\subsection*{Direct Proof}

\subsection*{Proof by Contraposition}

\subsection*{Proof by Contradiction}

\subsection*{Proof by Cases}

\newpage

\section*{Proofs}

A mathematical proof provides a means for guaranteeing that a statement is true. 

\bigbreak

So what is a proof? A proof is a finite sequence of steps, called \textbf{logical deductions}, which establishes the truth of a desired statement. In particular, the power of a proof lies in the fact hat using finite means, we can guarantee the truth of a statement with infinitely many cases.

\bigbreak

A proof is typically structured as follows:
\begin{itemize}
	\item[1.] Recall that there are certain statements, called \textbf{axioms} or \textbf{postulates}, that we accept without proof.
	\item[2.] From these axioms, a sequence of logical deductions follows.
	\item[3.] Each statement follows the previous one where each successive statement is necessarily true if the previous statement is true.
\end{itemize}

The rule of logic are a formal distillation of laws that were thought to underlie human thinking. They play a central role in the design of computers, starting with digital logic design or the fundamental principles behind the design of digital circuits.

\section*{Notation and Basic Facts}

Let $\mathbb{Z}$ denote the set of integers, i.e., $\mathbb{Z}=\{\dots,-2,-1,0,1,2,\dots\}$, and $\mathbb{N}$ the set of natural numbers $\mathbb{N}=\{0,1,2,\dots\}$.

\bigbreak

Recall that the sum or product of two integers is an integer, i.e., the set of integers is closed under addition and multiplication. The set of natural numbers is also closed under addition and multiplication.

\bigbreak

Given integers $a$ and $b$, we say $a$ divides $b$ (denoted $a\mid b$) iff there exists an integer $q$ such that $b=aq$. For example, $2\mid 10$ because there exists an intger $q=5$ such that $10=5\cdot 2$. We say a natural number $p\geq 2$ is prime if it is divisible only by 1 and itself.

\bigbreak

Finally, we use the notation $:=$ to indicate a definition. For example, $q:=6$ defines variable $q$ as having value 6.

\newpage

\section*{Direct Proof}

\textbf{Theorem 2.1.} $\textit{For any } a,b,c \in \mathbb{Z}, \textit{if } a\mid b \textit{ and } a\mid c, \textit{then } a\mid (b+c)$.

\begin{center}
	\begin{tabular}{|c|}
	\hline 
	\rule[-1ex]{0pt}{2.5ex} \textbf{Direct Proof}\\ 
	\rule[-1ex]{0pt}{2.5ex} Goal: To prove $P\Rightarrow Q$ \\ 
	\rule[-1ex]{0pt}{2.5ex} Approach: Assume $P$  \\ 
	\rule[-1ex]{0pt}{2.5ex} \vdots \\ 
	\rule[-1ex]{0pt}{2.5ex} Therefore $Q$ \\ 
	\hline
\end{tabular} 
\end{center}

For each $x$, the proposition we are trying to prove is of the form $P(x)\Rightarrow Q(x)$. A direct proof of this starts by assuming $P(x)$ for a generic value of $x$ and eventually concludes $Q(x)$ through a chain of implications.

\bigbreak

\textit{Proof of Theorem 2.1.} Assume that $a\mid b$ and $a\mid c$, i.e., there exist integers $q_1$ and $q_2$ such that $b=q_1a$ and $c=q_2a$. Then, $b+c=q_1a+q_2a=(q_1+q_2)a$. Since the $\mathbb{Z}$ is closed under addition, we conclude that $(q_1+q_2)\in\mathbb{Z}$, and so $a\mid(b+c)$, as desired.

\bigbreak

The key insight is that the proof did not assume any specific values for $a,b,$ and $c$; our proof holds for arbitrary $a,b,c\in\mathbb{Z}$. Thus, we have prove the desired claim.

\bigbreak

\textbf{Theorem 2.2.} \textit{Let} $0<n<1000$ \textit{be an integer. If the sum of the digits of n is divisible by 9, then n is divisible by 9}.

\bigbreak

Observe that this statement is equivalent to 
\begin{center}
	$(\forall n\in\mathbb{Z}^{+})(n<1000)\Rightarrow(\text{sum of }n\text{'s digits divisible by }9\Rightarrow n \text{ divisible by 9})$.
\end{center}
where $\mathbb{Z}^{+}$ denotes the set of positive integers, $\{1,2,\dots\}$. Now the proof proceeds similarly -- we start by assuming, for a generic value of $n$, that the sum of $n$'s digits is divisible by 9. Then we perform a sequence of implications to conclude that $n$ itself is divisible by 9.

\bigbreak

\textbf{Theorem 2.3} (Converse of Theorem 2.2)\textbf{.} \textit{Let} $0<n<1000$ \textit{be an integer. If $n$ is divisible by 9, then the sum of the digits of $n$ is divisible by 9}.

\bigbreak

\textit{Proof of Theorem 2.3.} Assume that $n$ is divisible by 9. We use the same notation for the digits of $n$ as we used in Theorem 2.2's proof.
\begin{center}
	\begin{align*}
		n \text{ is divisible by 9}&\Rightarrow n=9l \text{ for } \mathbb{Z} \\
		&\Rightarrow 100a+10b+c=9l \\
		&\Rightarrow 99a+9b+(a+b+c)=9l \\
		&\Rightarrow a+b+c=9l-99a-9b \\
		&\Rightarrow a+b+c=9(l-11a-b) \\
		&\Rightarrow a+b+c=9k \text{ for } k=l-11a-b\in\mathbb{Z}
	\end{align*}
\end{center}
We conclude that $a+b+c$ is divisible by 9.

\newpage

\section*{Proof by Contraposition}

Recall that any implication $P\Rightarrow Q$ is equivalent to its contrapositive $\neg Q\Rightarrow\neq P$. Sometimes $\neg Q\Rightarrow \neg P$ can be much simpler to prove than $P\Rightarrow Q$. Thus, a proof by contraposition proceeds by having $\neg Q\Rightarrow \neg P$ instead of $P \Rightarrow Q$.

\begin{center}
\begin{tabular}{|c|}
\hline 
\textbf{Proof by Contraposition} \\ 
Goal: To prove $P\Rightarrow Q$. \\ 
Approach: Assume $\neg Q$ \\ 
\vdots \\ 
Therefore $\neg P$ \\ 
Conclusion: $\neg Q\Rightarrow\neg P$, which is equivalent to $P\Rightarrow Q$. \\ 
\hline 
\end{tabular} 
\end{center}

\textbf{Theorem 2.4.} \textit{Let $n$ be a positive integer and let $d$ divide $n$. If $n$ is odd, then $d$ is odd}.

\bigbreak

\textit{Proof of Theorem 2.4.} We proceed by contraposition (If $d$ is even, then $n$ is even). If we assume $d$ is even, then, by definition, $d=2k$ for some $k\in\mathbb{Z}$. Because $d\mid n$, then $n=dl$ for some $l\in\mathbb{Z}$. Combining these two statements, we have $n=dl=(2k)l=2(kl)$. We conclude that $n$ is even.

\bigbreak

Note that the proof technique was stated as the first line. This is generally good practice.

\bigbreak

\textbf{Theorem 2.5} (Pigeonhole principle)\textit{.} \textit{Let $n$ and $k$ be positive integers. Place $n$ objects into $k$ boxes. If $n>k$, the at least one box must contain multiple objects}.

\bigbreak

\textit{Proof of Theorem 2.5.} We proceed by contraposition. If all boxes contain at most one object, then the number of objects is at most the number of boxes, i.e., $n\leq k$.

\section*{Proof by Contradiction}
The idea in a proof by contradiction relies crucially on the fact that if a proposition is not false, then it must be true.

\begin{center}
\begin{tabular}{|c|}
\hline 
\textbf{Proof by Contradiction} \\ 
Goal: To prove $P$. \\ 
Approach: Assume $\neg P$ \\ 
\vdots \\ 
$R$ \\ 
\vdots \\ 
$\neg R$ \\ 
Conclusion: $\neg P\Rightarrow\neg R\wedge R$, which is a contradiction. Thus, $P$. \\ 
\hline 
\end{tabular} 
\end{center}

A proof by contradiction shows that $\neg P\Rightarrow\neg R\wedge R\equiv$ False. The contrapositive of this statement is hence True $\Rightarrow P$.

\newpage

\textbf{Theorem 2.6.} \textit{There are infinitely many prime numbers}.

\bigbreak

Using a proof technique like direct proof seems to be very difficult. How would you construct infinitely many prime numbers? If we use contradiction instead by assuming the statement is false, i.e., there are only finitely many primes, bad things will happen.

\bigbreak

\textbf{Lemma 2.1.} \textit{Every natural number greater than one is either prime or has a prime divisor}.

\bigbreak

\textit{Proof of Theorem 2.6.} We proceed by contradiction. Suppose that Theorem 2.6 is false, i.e., there are only finitely many primes, say $k$ of them. Then, we can enumerate them: $p_1,p_2,p_3,\dots,p_k$.

\bigbreak

Define number $q:=p_1p_2p_3\dots p_k+1$, which is the product of all primes plus one. We can then claim that $q$ cannot be prime because by definition, it is larger than all the primes $p_1$ through $p_k$. By Lemma 2.1, we therefore conclude that $q$ has a prime divisor $p$. This will be our statement $R$.

\bigbreak

Next, because $p_1,p_2,p_3,\dots,p_k$ are all the primes, $p$ must be equal to one of them: thus, $p$ divides $r:=p_1p_2p_3\dots p_k$. Hence, $p\mid q$ and $p\mid r$, implying $p\mid(q-r)$. But $q-r=1$, implying $p\leq 1$, and hence $p$ is not prime; this is the statement $\neg R$. We thus have $R\wedge\neg R$, which is a contradiction, as desired.

\bigbreak

\textbf{Theorem 2.7.} \textit{$\sqrt{2}$ is irrational}.

\bigbreak

Why should contradiction be a good candidate proof technique to try here? Consider this: Theorem 2.6 and Theorem 2.7 share something fundamental -- in both cases, we wish to show something doesn't exist.

\bigbreak

For Theorem 2.6, we wished to show that a largest prime doesn't exist. \\
For Theorem 2.7, we wish to show that integers $a$ and $b$ satisfying $\sqrt{2}=a/b$ don't exist.

\bigbreak

\textbf{Lemma 2.2.} \textit{If $a^2$ is even, then $a$ is even}.

\bigbreak

\textit{Proof of Theorem 2.7.} We proceed y contradiction. Assume that $\sqrt{2}$ is rational. By the definition of rational numbers, there are integers $a$ and $b$ with no common factor other than 1, such that $\sqrt{2}=a/b$. Let our assertion $R$ state that $a$ and $b$ share no common factors.

\bigbreak

For any numbers $x$ and $y$, we know that $x=y\Rightarrow x^2=y^2$, hence $2=a^2/b^2$. Multiplying both sides by $b^2$, we have $a^2=2b^2$. Since $b$ is an integer, it follows that $b^2$ is an integer, and thus $a^2$ is even. Plugging in Lemma 2.2, we hence have that $a$ is even. In other words, there exists an integer $c$ such that $a=2c$.

\bigbreak

Combining all facts together, we have that $2b^2=4c^2$, or $b^2=2c^2$. Since $c$ is an integer, $c^2$ is an integer, and hence $b^2$ is even. Thus, again applying Lemma 2.2, we conclude that $b$ is even.

\bigbreak

But we have just shown that both $a$ and $b$ are even. In particular, this means they share the common factor 2. This implies $\neg R$. We conclude that $R\wedge\neg R$ holds; thus, we have a contradiction, as desired.

\newpage

\section*{Proof by Cases} 
\end{document}
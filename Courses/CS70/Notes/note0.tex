\documentclass[12pt,a4paper]{article}
\usepackage[utf8]{inputenc}
\usepackage[english]{babel}
\usepackage{amsmath}
\usepackage{amsfonts}
\usepackage{amssymb}
\usepackage{graphicx}
\usepackage[left=2cm,right=2cm,top=2cm,bottom=2cm]{geometry}
\author{Henry Yu}
\begin{document}

\raggedright

\begin{center}
	\huge{Note 0}
\end{center}

\section*{Review of Sets and Notation}

\textbf{Set}: well-defined collection of objects (\textbf{elements} or \textbf{members})

\begin{itemize}
	\item[--] usually denoted by capital letters
	\item[--] defined by listing its elements and surrounding the list by curly braces
\end{itemize}

\begin{center}
	$A=\{2, 3, 5, 7, 11\}$
\end{center}

If $x$ is an element of $A$, then we write $x \in A$. \\
If $y$ is not an element of $A$, then we write $y \notin A$. \\

\bigbreak

Two sets $A$ and $B$ are said to be \textbf{equal}, written as $A=B$, if they have the same elements.
\begin{itemize}
	\item[--] order and repetition of elements do not matter
	\begin{itemize}
		\item[--] $\{\text{red}, \text{white}, \text{blue}\}=\{\text{blue},\text{white},\text{red}\}=\{\text{red},\text{white},\text{white},\text{blue}\}$
	\end{itemize}
\end{itemize}

More complicated sets might be defined using a different notation. The set of all rational numbers, denoted by $\mathbb{Q}$, can be written as:
\begin{center}
	$\mathbb{Q}=\{\dfrac{a}{b}\mid a, b\text{ are integers}, b\neq0\}$
\end{center}

In English, this is read as "$\mathbb{Q}$ is the set of all fractions such that the numerator is an integer and the denominator is a non-zero integer."

\subsection*{Cardinality}
\textbf{Cardinality}: the size of a set

\bigbreak

If $A=\{1, 2, 3, 4\}$, then the cardinality of A, denoted by $\lvert A \rvert$, is 4. It is possible for the cardinality of a set to be 0. The \textbf{empty set}, denoted by the symbol $\emptyset$, is a unique such set.

\bigbreak

A set can also have an infinite number of elements, such as the set of all integers, prime numbers, or odd numbers.

\subsection*{Subsets and Proper Subsets}
If every element of a set $A$ is also in set $B$, then $A$ is a \textbf{subset} of $B$, written as $A\subseteq B$. Equivalently we can write $B\supseteq A$, or $B$ is a \textbf{superset} of $A$.

\bigbreak

A \textbf{proper subset} is a set $A$ that is strictly contained in set $B$, written as $A\subset B$, meaning that $A$ excludes at lest one element of $B$.

\bigbreak

Consider the set $B=\{1, 2, 3, 4, 5\}$. Then $A=\{1, 2, 3\}$ is both a subset and a proper subset of $B$, while $C=\{1, 2, 3, 4, 5\}$ is a subset but not a proper subset of $B$.

\newpage

Basic properties regarding subsets:
\begin{itemize}
	\item The empty set, denoted by $\{\}$ or $\emptyset$, is a proper subset of any nonempty set $A$: $\emptyset \subset A$
	\item The empty set is a subset of every set $B$: $\emptyset \subseteq B$
	\item Every set $A$ is a subset of itself: $A \subseteq A$
\end{itemize}

\subsection*{Intersections and Unions}
The \textbf{intersection} of a set $A$ with a set $B$, written as $A\cap B$, is the set containing all elements that are in both $A$ and $B$.

\bigbreak

Two sets are said to be \textbf{disjoint} if $A\cap B=\emptyset$.

\bigbreak

The \textbf{union} of the set $A$ with a set $B$, written as $A\cup B$, is the set of all elements which are in either $A$ or $B$ or both.

\bigbreak

If $A$ is the set of all positive even numbers, and $B$ is the set of all positive odd integers, then $A\cap B=\emptyset$, and $A\cup B=\mathbb{Z}^{+}$, or the set of all positive integers.

\bigbreak

Properties of intersections and unions:
\begin{itemize}
	\item $A\cup B=B\cup A$
	\item $A\cup \emptyset=A$
	\item $A\cap B = B\cap A$
	\item $A\cap \emptyset = \emptyset$
\end{itemize}

\subsection*{Complements}
If $A$ and $B$ are two sets, then the \textbf{relative complement} of $A$ in $B$, or the \textbf{set difference} between $B$ and $A$, written as $B-A$ or $B\setminus A$, is the set of elements in $B$, but not in\\
$A$: $B\setminus A=\{x\in B \mid x \notin A\}$.

\bigbreak

If $B=\{1, 2, 3\}$ and $A=\{3, 4, 5\}$, then $B\setminus A=\{1,2\}$.

\end{document}
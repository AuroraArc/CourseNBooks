
\documentclass[12pt,a4paper]{article}
\usepackage[utf8]{inputenc}
\usepackage[english]{babel}
\usepackage{amsmath}
\usepackage{amsfonts}
\usepackage{amssymb}
\usepackage{graphicx}
\usepackage[left=2cm,right=2cm,top=2cm,bottom=2cm]{geometry}
\author{Henry Yu}
\begin{document}

\raggedright

\begin{center}
	\huge{Note 0}
\end{center}

\section*{Review of Sets and Notation}

\textbf{Set}: well-defined collection of objects (\textbf{elements} or \textbf{members})

\begin{itemize}
	\item[--] usually denoted by capital letters
	\item[--] defined by listing its elements and surrounding the list by curly braces
\end{itemize}

\begin{center}
	$A=\{2, 3, 5, 7, 11\}$
\end{center}

If $x$ is an element of $A$, then we write $x \in A$. \\
If $y$ is not an element of $A$, then we write $y \notin A$. \\

\bigbreak

Two sets $A$ and $B$ are said to be \textbf{equal}, written as $A=B$, if they have the same elements.
\begin{itemize}
	\item[--] order and repetition of elements do not matter
	\begin{itemize}
		\item[--] $\{\text{red}, \text{white}, \text{blue}\}=\{\text{blue},\text{white},\text{red}\}=\{\text{red},\text{white},\text{white},\text{blue}\}$
	\end{itemize}
\end{itemize}

More complicated sets might be defined using a different notation. The set of all rational numbers, denoted by $\mathbb{Q}$, can be written as:
\begin{center}
	$\mathbb{Q}=\{\dfrac{a}{b}\mid a, b\text{ are integers}, b\neq0\}$
\end{center}

In English, this is read as ``$\mathbb{Q}$ is the set of all fractions such that the numerator is an integer and the denominator is a non-zero integer.''

\subsection*{Cardinality}
\textbf{Cardinality}: the size of a set

\bigbreak

If $A=\{1, 2, 3, 4\}$, then the cardinality of A, denoted by $\lvert A \rvert$, is 4. It is possible for the cardinality of a set to be 0. The \textbf{empty set}, denoted by the symbol $\emptyset$, is a unique such set.

\bigbreak

A set can also have an infinite number of elements, such as the set of all integers, prime numbers, or odd numbers.

\subsection*{Subsets and Proper Subsets}
If every element of a set $A$ is also in set $B$, then $A$ is a \textbf{subset} of $B$, written as $A\subseteq B$. Equivalently we can write $B\supseteq A$, or $B$ is a \textbf{superset} of $A$.

\bigbreak

A \textbf{proper subset} is a set $A$ that is strictly contained in set $B$, written as $A\subset B$, meaning that $A$ excludes at least one element of $B$.

\bigbreak

Consider the set $B=\{1, 2, 3, 4, 5\}$. Then $A=\{1, 2, 3\}$ is both a subset and a proper subset of $B$, while $C=\{1, 2, 3, 4, 5\}$ is a subset but not a proper subset of $B$.

\newpage

Basic properties regarding subsets:
\begin{itemize}
	\item The empty set, denoted by $\{\}$ or $\emptyset$, is a proper subset of any nonempty set $A$: $\emptyset \subset A$
	\item The empty set is a subset of every set $B$: $\emptyset \subseteq B$
	\item Every set $A$ is a subset of itself: $A \subseteq A$
\end{itemize}

\subsection*{Intersections and Unions}
The \textbf{intersection} of a set $A$ with a set $B$, written as $A\cap B$, is the set containing all elements that are in both $A$ and $B$.

\bigbreak

Two sets are said to be \textbf{disjoint} if $A\cap B=\emptyset$.

\bigbreak

The \textbf{union} of the set $A$ with a set $B$, written as $A\cup B$, is the set of all elements which are in either $A$ or $B$ or both.

\bigbreak

If $A$ is the set of all positive even numbers, and $B$ is the set of all positive odd integers, then $A\cap B=\emptyset$, and $A\cup B=\mathbb{Z}^{+}$, or the set of all positive integers.

\bigbreak

Properties of intersections and unions:
\begin{itemize}
	\item $A\cup B=B\cup A$
	\item $A\cup \emptyset=A$
	\item $A\cap B = B\cap A$
	\item $A\cap \emptyset = \emptyset$
\end{itemize}

\subsection*{Complements}
If $A$ and $B$ are two sets, then the \textbf{relative complement} of $A$ in $B$, or the \textbf{set difference} between $B$ and $A$, written as $B-A$ or $B\setminus A$, is the set of elements in $B$, but not in\\
$A$: $B\setminus A=\{x\in B \mid x \notin A\}$.

\bigbreak

If $B=\{1, 2, 3\}$ and $A=\{3, 4, 5\}$, then $B\setminus A=\{1,2\}$. \\
If $\mathbb{R}$ is the set of real numbers and $\mathbb{Q}$ is the set of rational numbers, then $\mathbb{R}\setminus\mathbb{Q}$ is the set of irrational numbers.

\bigbreak

Properties of complements:
\begin{itemize}
	\item $A\setminus A=\emptyset$
	\item $A\setminus \emptyset=A$
	\item $\emptyset\setminus A=\emptyset$
\end{itemize}

\newpage

\subsection*{Significant Sets}
In mathematics, some sets are referred to so commonly that they are denoted by special symbols. These include:
\begin{itemize}
	\item $\mathbb{N}$ denotes the set of all natural numbers: $\{0, 1, 2, 3, \dots\}$
	\item $\mathbb{Z}$ denotes the set of all integer numbers: $\{\dots, -2, -1, 0, 1, 2, \dots\}$
	\item $\mathbb{Q}$ denotes the set of all rational numbers: $\{\dfrac{a}{b}\mid a, b \in \mathbb{Z}, b \neq 0\}$
	\item $\mathbb{R}$ denotes the set of all real numbers
	\item $\mathbb{C}$ denotes the set of all complex numbers
\end{itemize}

\subsection*{Products and Power Sets}
The \textbf{Cartesian product} (also called the \textbf{cross product}) of two sets $A$ and $B$, written as $A\times B$, is the set of all pairs whose first component is an element of $A$ and whose second component is an element of $B$.

\bigbreak

In set notation, $A\times B=\{(a, b)\mid a\in A, b\in B\}$.

\bigbreak

If $A=\{1, 2, 3\}$ and $B=\{u, v\}$, then $A\times B=\{(1, u),\, (1, v),\, (2, u),\, (2, v),\, (3, u),\, (3, v)\}$. \\
$\mathbb{N}\times \mathbb{N}=\{(0, 0),\, (1, 0),\, (0, 1),\, (1, 1),\, (2, 0),\, \dots\}$ is the set of all pairs of natural numbers.

\bigbreak

Given a set $S$, the \textbf{power set} of $S$, denotes by $\mathcal{P}(S)$, is the set of all subsets of $S$: $\{T\mid T\subseteq S\}$.

\bigbreak

If $\lvert S \rvert = k$, then $\lvert \mathcal{P}(S)\rvert=2^k$.

\section*{Mathematical Notation}
\subsection*{Sums and Products}
There is a compact notation for writing sums or products of large numbers of items.

\bigbreak

To write $1+2+\dots+n$, we write $\sum_{i=1}^{n}i$. More generally we can write the sum $f(m)+f(m+1)+\dots+f(n)$ as $\sum_{i=m}^{n}f(i)$. For example $\sum_{i=5}^{n}i^2=5^2+6^2+\dots+n^2$.

\bigbreak

To write the product $f(m)f(m+1)+\dots+f(n)$, we use the notation $\prod_{i=m}^{n}f(i)$. For example, $\prod_{i=1}^{n}i=1\cdot 2\cdots n$ is the product of the first $n$ positive integers.

\subsection*{Universal and Existential Quantifiers}
$\forall$: universal quantifier (``for all'') \\
$\exists$: existential quantifier (``there exists'')

\bigbreak

Consider the statement: For all natural numbers $n$, $n^2+n+41$ is prime. Here, $n$ is quantified to any element of the set $\mathbb{N}$ of natural numbers. \\
In notation, we write $(\forall n \in \mathbb{N})(n^2+n+41$ is prime).

\bigbreak

$n^2+n+41$ is indeed prime for small values of $n$, but there are values that make it not prime ($n=41$). Therefore, this statement is not true.

\bigbreak

Consider this new statement: $(\exists x \in \mathbb{Z})(x<2 \text{ and } x^2=4)$. The statement says that there is an integer $x$ which is less than 2, but its square is equal to 4. This statement is true ($x=-2$).

\bigbreak

Statements an be written using both kinds of quantifiers:
\begin{itemize}
	\item[1.] $(\forall x \in \mathbb{Z})(\exists y \in \mathbb{Z})(y>x)$
\end{itemize}
For all $x$ in the set of integers, there exists a $y$ in the set of integers where $y$ is greater than $x$ (true).

\begin{itemize}
	\item[2.] $(\exists y \in \mathbb{Z})(\forall x \in \mathbb{Z})(y>x)$
\end{itemize}
There exists a $y$ in the set of integers which is greater than $x$ for all $x$ in the set of integers (false).
\end{document}
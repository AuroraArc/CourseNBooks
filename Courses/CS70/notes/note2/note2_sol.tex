\documentclass[12pt,a4paper]{article}
\usepackage[utf8]{inputenc}
\usepackage[english]{babel}
\usepackage{amsmath}
\usepackage{amsfonts}
\usepackage{amssymb}
\usepackage{amsthm}
\usepackage{graphicx}
\usepackage{xcolor}
\usepackage[left=2cm,right=2cm,top=2cm,bottom=2cm]{geometry}


\newtheorem{theorem}{Theorem}[section]
\newtheorem{lemma}{Lemma}[section]
\theoremstyle{definition}
\newtheorem{defn}{Definition}[section]
\setcounter{section}{2}
\setcounter{theorem}{8}
\setcounter{lemma}{1}

\begin{document}

\raggedright

\section*{Note 2 Solutions}

1. Generalize the proof of Theorem 2.2 so that it works for any positive integer $n$. (Hint: Suppose $n$ has $k$ digits, and write $a_i$ for the digits of $n$, so that $n=\sum_{i=0}^{k-1}(a_i\cdot 10^i)$.)

\begin{theorem}
Let $n\in\mathbb{Z}^+$. If the sum of the digits of n is divisible by 9, then $n$ is divisible by 9.
\end{theorem}

\kern-1em
\color{red}
\begin{proof}[Proof of Theorem 2.9]
We proceed with direct proof. Suppose $n$ has $k$ digits and $n=\sum_{i=0}^{k-1}(a_i\cdot 10^i)$. $i$ goes all the way up to $k-1$ since we start on index 0. We can rewrite $n$ as
\begin{center}
$n=10^0\cdot a_0+10^1\cdot a_1+10^2\cdot a_2 + \cdots + 10^{k-2}\cdot a_{k-2} + 10^{k-1}\cdot a_{k-1}$.
\end{center}

Assume that the sum of the digits of $n$ is divisible by 9, i.e.

\begin{equation}\label{eq1}
\exists l\in\mathbb{Z} \text{ such that } \sum_{i=0}^{k-1}a_i=9l.
\end{equation}

Adding $n=\sum_{i=1}^{k-1}(a_i\cdot [10^i-1])$ to both sides of Equation \eqref{eq1}, we have

\begin{align*}
&a_0+10a_1+100a_2\cdots+10^{k-1}a_{k-1}=n=9l+9a_1+99a_2+\cdots+(10^{k-1}-1)a_{k-1} \\
&=9(l+a_1+9a_2+\cdots+[10^{k-2}+10^{k-3}+\cdots+1]a_{k-1}).
\end{align*}

We conclude that $n$ is divisible by 9.
\end{proof}

\color{black}
2. Prove Lemma 2.2. (Hint: First try a direct proof. Then, try contraposition. Which proof approach is better suited to proving this lemma?)

\begin{lemma}
If $a^2$ is even, then $a$ is even.
\end{lemma}

\kern-1em
\color{red}
\begin{proof}[Proof of Lemma 2.2 using direct proof]
We proceed with direct proof. One thing to point out is that non-integer numbers cannot be even nor odd, so $a$ and $a^2$ must be whole numbers, therefore meaning that whatever $a^2$ equals must be a perfect square. Assume that $a^2$ is even, i.e., 

\begin{center}
$\exists k\in\mathbb{Z}^+ \text{ such that } a^2=(2k)^2.$
\end{center}

Taking the square root of both sides yields 

\begin{center}
$a=2k.$
\end{center}

Since $k$ is divisible by 2, it holds that $a$ is even, as desired.
\end{proof}

\begin{proof}[Proof of Lemma 2.2 using contraposition]
We proceed by contraposition. This means that if $a$ is not even, then $a^2$ is not even. Assume that $a$ is not even, i.e.,

\begin{center}
$\exists k\in\mathbb{Z} \text{ such that } a=2k+1.$
\end{center}

Squaring both sides yields

\begin{center}
$a^2=4k^2+4k+1=2(2k^2+2k)+1$,
\end{center}

which is not even since there isn't a factor of two on the entire right side. Therefore $a^2$ is odd, as desired.
\end{proof}

\end{document}
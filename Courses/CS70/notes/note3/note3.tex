\documentclass[12pt,a4paper]{article}
\usepackage[utf8]{inputenc}
\usepackage[english]{babel}
\usepackage{amsmath}
\usepackage{amsfonts}
\usepackage{amssymb}
\usepackage{amsthm}
\usepackage{graphicx}
\usepackage[left=2cm,right=2cm,top=2cm,bottom=2cm]{geometry}


\newtheorem{theorem}{Theorem}[section]
\newtheorem{lemma}{Lemma}[section]
\theoremstyle{definition}
\newtheorem{defn}{Definition}[section]
\setcounter{section}{3}

\begin{document}

\raggedright

\begin{center}
\huge{Note 3}
\end{center}

\section*{Mathematical Induction}
Induction is a powerful tool which is used to establish that a statement holds  for \textit{all} natural numbers. Of course, there are infinitely many natural numbers -- induction provides a way to reason about them by finite means.

\bigbreak

Suppose we wish to prove the statement: For all natural numbers $n$, $0+1+2+3+\cdots+n$ $=\frac{n(n+1)}{2}$. More formally, we can write this as

\begin{equation}\label{eq1}
\forall n\in\mathbb{N}, \sum_{i=0}^{n}i=\frac{n(n+1)}{2}.
\end{equation}

How would you prove this? You could begin by checking that it holds for $n=0, 1, 2$, and so forth, but there's an infinite number of values of $n$ for which it needs to be checked. Moreover, checking just the first few values of $n$ does not suffice to conclude the statement holds for all $n\in\mathbb{N}$.

\bigbreak

Consider this statement that was shown in a previous note: $\forall n\in\mathbb{N}, n^2-n+41$ is a prime number. Check that it holds for the first few natural numbers. Now check the case for $n=41$.

\bigbreak

In mathematical induction, we instead make an interesting observation: Suppose the statement holds for some value $n=k$, i.e., $\sum_{i=0}^{k}i=\frac{k(k+1)}{2}$. (This is called the \textit{induction hypothesis.} Then:

\begin{equation}\label{eq2}
\left(\sum_{i=0}^{k}i\right)+(k+1)=\frac{k(k+1)}{2}+(k+1)=\frac{(k+1)(k+2)}{2},
\end{equation}

i.e., the claim also holds for $n=k+1$! In other words, if the statement holds for some $k$, then it must also hold for $k+1$. Let us call the argument above the \textit{inductive step}. If we can show that the statement holds for $k$, then the inductive step allows us to conclude that it also holds for $k+1$; but that it holds for $k+1$, the inductive step implies that it holds for $k+2$; and so on.

\bigbreak

So have we proven Equation \eqref{eq1}? Not yet. The problem is that in order to apply the inductive step, we first have to establish that Equation \eqref{eq1} holds for some initial value of $k$. Since our aim is to prove the statement for all natural numbers, the obvious choice is $k=0.$ We call this choice of $k$ the \textit{base case}. Then, if the base case holds, the axiom of mathematical induction says that the inductive step allows us to conclude that Equation \eqref{eq1} indeed holds for all $n\in\mathbb{N}$.

\bigbreak

Let's formally rewrite this proof.

\begin{theorem}
$\displaystyle\forall n\in\mathbb{N},\sum_{i=0}^{n}i=\frac{n(n+1)}{2}.$
\end{theorem}

\begin{proof}[Proof of Theorem 3.1]
We proceed by induction on the variable $n$.

\bigbreak

\textit{Base case (n=0)}: Here, we have $\displaystyle\sum_{i=0}^{0}i=0=\frac{0(0+1)}{2}.$ Thus, the base case is correct.

\bigbreak

\textit{Induction Hypothesis}: For arbitrary $n=k\geq0$, assume that $\displaystyle\sum_{i=0}^{k}i=\frac{k(k+1)}{2}$. In words, the induction hypothesis says ``let's assume we have proved the statement for an arbitrary value of $n=k\geq0.$''

\bigbreak

\textit{Inductive Step}: Prove the statement for $n=(k+1)$, i.e., show that $\displaystyle\sum_{i=0}^{k+1}i=\frac{(k+1)(k+2)}{2}$:

\begin{equation}\label{eq3}
\sum_{i=0}^{k+1}i=\left(\sum_{i=0}^{k}i\right)+(k+1)=\frac{k(k+1)}{2}+(k+1)=\frac{k(k+1)+2(k+1)}{2}=\frac{(k+1)(k+2)}{2},
\end{equation}

where the second equality follows from the induction hypothesis. By the principle of mathematical induction, the claim follows.
\end{proof}

\textbf{Recap.} \\
\begin{enumerate}
\item \textbf{Base Case}: Prove that $P(0)$ is true. \\
\item \textbf{Induction Hypothesis}: For arbitrary $k\geq0$, assume that $P(k)$ is true. \\
\item \textbf{Inductive Step}: With the assumption of the induction hypothesis in hand, show that $P(k+1)$ is true.
\end{enumerate}

\bigbreak

Finally, a word about choosing a base case -- in general, the choice of base case will naturally depend on the claim you with to prove.

\bigbreak

Let us do another proof by induction. Recall that for integers $a$ and $b$, we say that $a$ divides $b$, denoted $a\mid b$, iff there exists an integer $q$ satisfying $b=aq$.

\begin{theorem}
For all $n\in\mathbb{N}, n^3-n$ is divisible by 3.
\end{theorem}


\end{document}
\documentclass[12pt,a4paper]{article}
\usepackage[utf8]{inputenc}
\usepackage[english]{babel}
\usepackage{amsmath}
\usepackage{amsfonts}
\usepackage{amssymb}
\usepackage{graphicx}
\usepackage[left=2cm,right=2cm,top=2cm,bottom=2cm]{geometry}
\author{Henry Yu}
\begin{document}

\raggedright

\begin{center}
	\huge{Note 1}
\end{center}

\section*{Propositional Logic}

To become fluent in working with mathematical statements, you need to understand the basic framework of the language of mathematics.

\bigbreak

\textbf{Proposition}: a statement which is either true or false

\bigbreak

These statements are propositions:
\begin{itemize}
	\item[1.] $\sqrt{3}$ is irrational.
	\item[2.]$1+1=5$.
\end{itemize}

These statements are not propositions:
\begin{itemize}
	\item[1.] $2+2$.
	\item[2.]$x^2+3x=5$. (since x is unknown)
\end{itemize}

Propositions should not include fuzzy terms, so these statements aren't propositions either (although some sources may say they are):
\begin{itemize}
	\item[1.] Arnold Schwarzenegger often eats broccoli. ("often" is fuzzy)
	\item[2.] Henry VIII was unpopular. ("unpopular" is fuzzy)
\end{itemize}

Propositions may be joined together to form more complex statements. Let $P$, $Q$, \text{and } $R$ be variables representing propositions. The simplest way of joining these propositions together is to use the connectives "and," "or," and "not."
\begin{itemize}
	\item[1.] \textbf{Conjunction}: $P\wedge Q$ ("$P$ and $Q$"). True only when both $P$ and $Q$ are true.
	\item[2.] \textbf{Disjunction}: $P\vee Q$ ("$P$ or $Q$"). True when at least one of $P$ and $Q$ is true.
	\item[3.] \textbf{Negation}: $\neg P$ ("not $P$"). True when $P$ is false.
\end{itemize}

Statements with variables (like these) are called \textbf{propositional forms}.

\bigbreak

A fundamental principle known as the \textbf{law of the excluded middle} says that, for any proposition $P$, either $P$ is true or $\neg P$ is true (but not both). Thus $P\vee\neg P$ is always true, regardless of the truth value of $P$. A propositional form that is always true regardless of its truth values is called a \textbf{tautology}. Conversely, a statement such as $P\wedge\neg P$, which is always false, is called a \textbf{contradiction}.

\bigbreak

A \textbf{truth table} is used to describe the possible values of a propositional form. Truth tables are the same as function tables: you list all possible input values for the variables, and then list the outputs given for those inputs.

\bigbreak

Here are the truth tables for conjunction, disjunction, and negation:
\begin{center}
	\begin{tabular}{ |c|c| }
		\hline
		$P$ & $Q$ \\
		\hline
		T & T \\
		\hline
		T & F \\ 
		\hline
		F & T \\ 
		\hline
		F & F \\
		\hline
	\end{tabular}
	\begin{tabular}{ |c| }
		\hline
		$P\wedge Q$ \\
		\hline
		T \\
		\hline
		F \\
		\hline
		F \\
		\hline
		F \\
		\hline
	\end{tabular}
	\begin{tabular}{ |c| }
		\hline
		$P\vee Q$ \\
		\hline
		T \\
		\hline
		T \\
		\hline
		T \\
		\hline
		F \\
		\hline
	\end{tabular}
	\begin{tabular}{ |c|c| }
		\hline
		$P$ & $\neg P$ \\
		\hline
		T & F \\
		\hline
		F & T \\
		\hline		
	\end{tabular}
\end{center}

\newpage

The most important and subtle propositional form is an \textbf{implication}:
\begin{itemize}
	\item[4.] \textbf{Implication}: $P\Rightarrow Q$ ("$P$ implies $Q$"). This is the same as "if $P$, then $Q$."
\end{itemize}

Here, $P$ is called the \textbf{hypothesis} of the implication, and $Q$ is the \textbf{conclusion}.

\bigbreak

Here are some examples:
\begin{itemize}
	\item If you stand in the rain, then you'll get wet.
	\item If you passed the class, you received a certificate.
\end{itemize}

An implication $P\Rightarrow Q$ is false only when $P$ is true and $Q$ is false.

\bigbreak

Here is the truth table for $P\Rightarrow Q$ (along with an additional column):
\begin{center}
	\begin{tabular}{ |c|c|c|c| }
		\hline
		$P$ & $Q$ & $P\Rightarrow Q$ & $\neg P \vee Q$ \\
		\hline
		T & T & T & T \\
		\hline
		T & F & F & F \\
		\hline
		F & T & T & T \\
		\hline
		F & F & T & T \\ 
		\hline
	\end{tabular}
\end{center}

Note the $P\Rightarrow Q$ is always true when $P$ is false. When an implication is stupidly true because the hypothesis is false, we say that it is \textbf{vacuously true}.

\bigbreak

Note also that $P\Rightarrow Q$ is \textbf{logically equivalent} to $\neg P \vee Q$. We write this as $(P\Rightarrow Q)\equiv(\neg P \vee Q)$.

\bigbreak

Here are some different ways of meaning $P\Rightarrow Q$:
\begin{itemize}
	\item[1.] if $P$, then $Q$
	\item[2.] $Q$ if $P$
	\item[3.] $P$ only if $Q$
	\item[4.] $P$ is sufficient for $Q$
	\item[5.] $Q$ is necessary for $P$
	\item[6.] $Q$ unless not $P$
\end{itemize}

If both $P\Rightarrow Q$ and $Q\Rightarrow P$ are true, then we say "$P$ if and only if $Q$" (abbreviated "$P$ iff $Q$"). Formally, we write $P\Leftrightarrow Q$. Note that $P\Leftrightarrow Q$ is true when $P$ and $Q$ have the same truth values (both true or false).

\bigbreak

Given an implication $P\Rightarrow Q$, we can also define its
\begin{itemize}
	\item[1.] \textbf{Contrapositive}: $\neg Q\Rightarrow \neg P$
	\item[2.] \textbf{Converse}: $Q\Rightarrow P$
\end{itemize}

\newpage

Here are some truth tables:

\begin{center}
	\begin{tabular}{ |c|c|c|c||c|c|c|c| }
		\hline
		$P$ & $Q$ & $\neg P$ & $\neg Q$ & $P\Rightarrow Q$ & $Q\Rightarrow P$ & $\neg Q\Rightarrow \neg P$ & $P\Leftrightarrow Q$ \\
		\hline
		T & T & F & F & T & T & T & T \\
		\hline
		T & F & F & T & F & T & F & F \\
		\hline
		F & T & T & F & T & F & T & F \\
		\hline
		F & F & T & T & T & T & T & T \\
		\hline
	\end{tabular}
\end{center}

Note that $P\Rightarrow Q$ and its contrapositive have the same truth values, so they are logically equivalent: $(P\Rightarrow Q)\equiv(\neg Q\Rightarrow\neg P)$.

\bigbreak

Also note that $P\Rightarrow Q$ and $Q\Rightarrow P$ are not logically equivalent.

\section*{Quantifiers}

The mathematical statements you'll see in practice will look something like this:
\begin{itemize}
	\item[1.] For all natural numbers $n$, $n^2+n+41$ is prime.
	\item[2.] If $n$ is an odd integer, so is $n^3$.
	\item[3.] There is an integer $k$ that is both even and odd.
\end{itemize}

These statements assert something about lots of simple propositions all at once. For instance, the first statement is asserting that $0^2+0+41$ is prime, $1^2+1+41$ is prime, and so on.

\bigbreak

The last statement says that as $k$ ranges over all possible integers, we will find at least one value of $k$ for which the statement is satisfied.

\bigbreak

Compared to the previous statement "$x^2+3x=5$," these examples are quantified over a "universe." To express these statements mathematically, we need two \textbf{quantifiers}:
\begin{itemize}
	\item[1.] the universal quantifier $\forall$ ("for all")
	\item[2.] the existential quantifier $\exists$ ("there exists")
\end{itemize}

Examples:
\begin{itemize}
	\item[1.] "Some mammals lay eggs."
\end{itemize}

Mathematically, "some" means "at least one," so the statement is actually saying "There exists a mammal $x$ such that $x$ lays eggs." If we let our universe $U$ to be the set of mammals, then we can write: $(\exists x\in U)(x \text{ lays eggs})$.

\begin{itemize}
	\item[2.] "For all natural numbers $n$, $n^2+n+41$ is prime."
\end{itemize}

In this case, our universe becomes the set of natural numbers, denoted as $\mathbb{N}$: $(\forall n \in \mathbb{N})(n^2+n+41\text{ is prime})$.

\bigbreak

We refer to a statement which refers to a variable as a \textbf{predicate} or as a \textbf{propositional formula} when replacing the variable with a value makes the statement true or false.

\newpage

Note that in a finite universe, we can express existentially and universally quantified propositions without quantifiers, using disjunctions and conjunctions respectively.

\bigbreak

For example, if our universe $U$ is $\{1,2,3,4\}$, then \\
$(\exists x \in U)P(x)$ is logically equivalent to $P(1)\vee P(2)\vee P(3)\vee P(4)$, and \\
$(\forall x \in U)P(x)$ is logically equivalent to $P(1)\wedge P(2) \wedge P(3) \wedge P(4)$.
 
\bigbreak
 
However, in an infinite universe, this is not possible.

\section*{Negation}

Let's look at how to negate conjunctions and disjunctions:
\begin{center}
 	$\neg(P\wedge Q)\equiv(\neg P\vee\neg Q)$ \\
 	$\neg(P\vee Q)\equiv(\neg P\wedge\neg Q)$
\end{center}

These two equivalences are known as \textbf{De Morgan's Laws}, and they are quite intuitive: for example, if it is not the case that $P\wedge Q$ is true, then either $P$ or $Q$ is false (and vice versa).

\bigbreak

Negating propositions involving quantifiers actually follows analogous laws.

\bigbreak

Assume that the universe is $\{1,2,3,4\}$ and let $P(x)$ denote the propositional formula "$x^2>10$". \\
Check that $\exists xP(x)$ is true but $\forall xP(x)$ is false. \\
Observe that both $\neg(\forall xP(x))$ and $\exists x\neg P(x)$ are true because $P(1)$ is false. \\
Also note that both $\forall x\neg P(x)$ and $\neg(\exists xP(x))$ are false, because $P(4)$ is true.

\bigbreak

The fact that each pair of statements had the same truth value, as the equivalences
\begin{center}
	$\neg(\forall xP(x))\equiv\exists x\neg P(x)$ \\
	$\neg(\exists xP(x))\equiv \forall x\neg P(x)$
\end{center}
are laws that hold for any proposition $P$ quantified over any universe (including infinite ones).

\bigbreak

It is helpful to think of English sentences to convince yourself (informally) that these laws are true.

\bigbreak

For example, assume that we are working within the universe $\mathbb{Z}$ (the set of all integers), and that $P(x)$ is the proposition "$x$ is odd." We know that the statement $(\forall xP(x))$ is false, since not every integer is odd.

\bigbreak

Therefore, we expect its negation, $\neg(\forall xP(x))$, to be true. How would you say the negation in English? If it is not true that every integer is odd, then there must exist some integer which is not odd (i.e., even). $(\exists x\neg P(x))$

\bigbreak

To see a more complex example, fix some universe and propositional formula $P(x,y)$. Assume we have the proposition $\neg(\forall x\exists yP(x,y))$ and we want to push the negation operator inside the quantifiers. By the above laws, we can do so:
\begin{center}
	$\neg(\forall x\exists yP(x,y))\equiv\exists x\neg(\exists yP(x,y))\equiv\exists x\forall y\neg P(x,y)$
\end{center}

Notice that we broke the complex negation into a smaller, easier problem as the negation propagated itself through the quantifiers. Note also that the quantifiers "flip" as we go.

\newpage

Write the sentence "there are at least three distinct integers $x$ that satisfy $P(x)$" as a proposition using quantifiers. One way to do it is
\begin{center}
	$\exists x\exists y\exists z(x\neq y\wedge y\neq z\wedge x\neq z\wedge P(x) \wedge P(y)\wedge P(z))$
\end{center}

Write the sentence "there are \textbf{at most} three distinct integers $x$ that satisfy $P(x)$" as a proposition using quantifiers. One way to do it is
\begin{center}
	$\exists x\exists y\exists z\forall d(P(d)\Rightarrow d=x\vee d=y\vee d=z)$
\end{center}

Here is an equivalent way to do it:
\begin{center}
	$\forall x\forall y\forall v\forall z((x\neq y\wedge y\neq v \wedge v\neq x\wedge x\neq z\wedge y\neq z \wedge v\neq z)\Rightarrow\neg(P(x)\wedge P(y)\wedge P(v)\wedge P(z)))$
\end{center}

What if we want to express the sentence "there are \textbf{exactly} three distinct integers $x$ that satisfy $P(x)$?" We can use the conjunction of the two propositions above.
\end{document}
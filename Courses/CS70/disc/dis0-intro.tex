\documentclass[12pt,a4paper]{article}
\usepackage[utf8]{inputenc}
\usepackage[english]{babel}
\usepackage{amsmath}
\usepackage{amsfonts}
\usepackage{amssymb}
\usepackage{graphicx}
\usepackage{xcolor}
\usepackage[left=2cm,right=2cm,top=2cm,bottom=2cm]{geometry}
\begin{document}

\raggedright

\begin{center}
\huge{Dis 00}
\end{center}

\section{Implication}

Which of the following implications are always true, regardless of $P$? Give a counterexample for each false assertion (i.e. come up with a statement $P(x,y)$ that would make the implication false).

\bigbreak

(a) $\forall x\forall yP(x,y)\Rightarrow\forall y\forall xP(x,y)$

\bigbreak

\color{red}
The implication is true because quantifiers of the same type can be switched around (i.e. they are commutative): 

\begin{center}
$\forall x\forall yP(x,y)\equiv \forall y\forall xP(x,y) \text{ and } \exists x\exists yP(x,y)\equiv \exists y\exists xP(x,y)$
\end{center}

\bigbreak

\color{black}
(b) $\forall x\exists yP(x,y) \Rightarrow \exists y\forall xP(x,y)$

\bigbreak

\color{red}
The implication is false. Let $P(x,y)$ be $x=y$; then it is easy how the antecedent is true. For any $x$ chosen, there exists a $y$ that equals $x$, which is true. However, there does not exist one value of $y$ that can equal all values of $x$.

\bigbreak

\color{black}
(c) $\exists x\forall yP(x,y)\Rightarrow\forall y\exists xP(x,y)$

\bigbreak

\color{red}
The implication is true. The first statement says that there exists an $x$, where $x_1$ for every $y$, is true. Setting $x=x_1$ (the second $x$ in the implication) means that for every $y$, there is an $x$ that makes the second statement true.

\color{black}
\section{Equivalences with Quantifiers}

\section{XOR}


\end{document}